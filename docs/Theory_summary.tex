\documentclass{article}
\usepackage{amsmath, amssymb}

\title{Theoretical Foundations of Pytron-QFT}
\author{DeepSeek-AI}

\begin{document}
\maketitle

\section{Introduction}
Pytron-QFT bridges quantum field theory (QFT) and deep learning. It represents data as excitations in a learnable quantum field:
\[
\phi(x) = \int \frac{d^3p}{(2\pi)^3} \frac{1}{\sqrt{2E_p}} \left( a_p e^{-ip\cdot x} + a_p^\dagger e^{ip\cdot x} \right)
\]
where creation/annihilation operators $a_p^\dagger, a_p$ are replaced by learnable parameters.

\section{Quantum Field Representation}
Data $x\in\mathcal{X}$ is embedded as:
\[
\psi(x) = \mathcal{N} \exp\left(-\int d^d y\, \phi(y) K(x,y) \right) |0\rangle
\]
with learnable kernel $K$ and normalization $\mathcal{N}$.

\section{Evolution Dynamics}
The system evolves under a learnable Hamiltonian density:
\[
\mathcal{H} = \pi^2 + |\nabla\phi|^2 + m^2\phi^2 + \lambda \phi^4
\]
with renormalized parameters $m, \lambda$.

\section{Path Integral Optimization}
Parameters are optimized via Feynman path integrals:
\[
\langle \mathcal{O} \rangle = \frac{\int \mathcal{D}\theta\, \mathcal{O}(\theta) e^{-\beta S[\theta]}}{\int \mathcal{D}\theta\, e^{-\beta S[\theta]}}
\]
where $S$ is the action functional of parameters.

\section{Topological Features}
Persistent homology of wavefunction $|\psi(x)|^2$ provides regularization:
\[
\mathcal{L}_{topo} = \sum_i \beta_i \log(\text{persistence}_i)
\]
where $\beta_i$ are Betti numbers.

\section{Conclusion}
Pytron-QFT offers a unified framework for quantum-inspired machine learning with strong theoretical foundations in QFT.
\end{document}